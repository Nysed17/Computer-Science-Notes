\documentclass{article}
\usepackage{minted}

\title{Programmazione ad Oggetti}
\author{Alex Narder}
\date{\today}

\begin{document}
\maketitle

\section{Definizioni}

\begin{itemize}
   \item[•] \textbf{Variabili}: Aree di memoria a cui viene assegnato un valore
   \item[•] \textbf{Classi}: Una classe è la definizione di un tipo di oggetto, quindi al suo interno contiene metodi, campi e variabili 
   \item[•] \textbf{Oggetto}: Un oggetto di una classe è un'istanza della classe, e ha tutti gli attributi di quella classe
   \item[•] \textbf{Costruttori}: Il costruttore è quel metodo di una classe il cui compito è proprio quello di creare nuove istanze. Il costruttore è dichiarato 
      con il nome della classe e serve soprattutto per inizializzare i campi con un valore.
   \item[•] \textbf{Metodi}: Essenzialmente sono le funzioni contenute nella classe
   \item[•] \textbf{Campi}: Anche definiti fields, sono le variabili della classe che vengono dichiarate al suo interno ma fuori dai costruttori e dai metodi 
\end{itemize}
\newpage
\begin{minted}{java}
   class Appunti{
      int n;               //Campo (field)
      int res;             //Campo (field)

      Appunti(int n){      //Costruttore con valore 
         this.n = n;       //Inizializzo n della classe assegnandogli valore di n
      }   

      void calcRes(){      //Questo è un metodo per calcolare il risultato finale
         res = n*n;
      }

      void print(){        //Metodo, e questa è la sua intestazione 
         System.out.println("The result is:");  //Corpo del metodo
      }

      int returnRes(){     //Metodo
         print();          //Invocazione del metodo print()
         return res;
      }
   }


   class Nuova{
      public static void main(String[] args){
         int n = 5;
         Appunti newObj = new Appunti(n);    //Nuovo oggetto
      
      }
   }
\end{minted}

\newpage 

\section{Javadoc}

Ci sono diversi modi per commentare in Java, e uno dei pregi di questo linguaggio è 
quello di poter integrare la documentazione con il codice stesso:\\
\\
Formato dei commenti:
\\
• $/*$ $commenti$ $*/$
\\
• $//$ $commenti$
\\
• $/**$ $commenti$ $documentazione$ $*/$
\\
\\
Quest'ultimo genera automaticamente la documentazione in formato HTML utilizzando il 
programma javadoc (scritto in Java).\\
\\
Il commento ha due parti:
\\
una descrizione a parole, seguita da alcuni \textbf{block tags}, ovvero delle etichette standard che discutono alcuni aspetti.
\\
\begin{minted}{java}
/**
   * Accede un elenco di studenti di un url e ritorna uno
   * Studente che ha come matricola il numero passato come
   * parametro. L'url deve essere un {@link URL} assoluto.
   * <p> <!-- -->
   * Il metodo ritorna sempre un valore. Se lo studente
   * non esiste o l’elenco è vuoto, ritorna null.
   * @param url   URL assoluto dove si trova l'elenco
   * @param matr  la matricola dello studente da trovare
   * @returns     il link allo Studente
   * @see         Studente
*/
public Studente getStudente(URL url, int matricola)
{ try {...}
   catch(MalformedUrlException) {return null;}
   catch(ElencoVuotoException) {return null;} ...
}
\end{minted}
Una volta scritta la documentazione, questa viene generata con il programma \textbf{javadoc}.
Ad esempio se abbiamo definito una classe \textbf{Dado} nel file \textbf{Dado.java} possiamo
generare la documentazione con \textbf{javadoc Dado.java}.
\\
Vengono generati i file di documentazione della classe a partire da un \textbf{index.html} che 
possono essere visualizzati con un browser.

\newpage

\subsection{Block Tag comuni}

\begin{itemize}
   \item[•] \textbf{@author}: specifica il nome dell'autore, viene considerato solo se javadoc viene eseguito con l'opzione -author 
   \item[•] \textbf{@version}: indica un numero di versiome, viene considerato solo se javadoc viene eseguito con l'opzione -version 
   \item[•] \textbf{@param}: descrive uno dei parametri passati
   \item[•] \textbf{@returns}: descrive il valore di ritorno restituito al chiamante 
   \item[•] \textbf{@throws}: (per ogni eccezione si può verificare) descrive il tipo di eccezione e la sua descrizione 
   \item[•] \textbf{@see}: rimanda a un'altra voce di documentazione 
   \item[•] \textbf{@deprecated}: indica che non andrebbe più usato
\end{itemize}

\subsection{Documentazione iniziale}
\begin{minted}{java}
/**
   * This class contains various methods for manipulating arrays (such as
   * sorting and searching). This class also contains a static factory
   * that allows arrays to be viewed as lists.
   *
   * @author Josh Bloch
   * @author Neal Gafter
   * @author John Rose
   * @since 1.2
*/
\end{minted}

\subsection{Documentazione di un field}
Nulla di che, solo un commento con la sua descrizione:
\\
\begin{minted}{java}
/**
   * The minimum array length below which a parallel sorting
   * algorithm will not further partition the sorting task. Using
   * smaller sizes typically results in memory contention across
   * tasks that makes parallel speedups unlikely.
*/
private static final int MIN_ARRAY_SORT_GRAN = 1 << 13;
\end{minted}

\newpage

\subsection{Documentazione di un metodo}
Un metodo viene definito con i tag \textbf{@return, @param <nome>, @throw, @deprecated}:
\\
\begin{minted}{java}
/**
   * Finds and returns the index of the first mismatch
     between two
   * {@code double} arrays, otherwise return -1 if no
     mismatch is found.
   * (...)
   * @param a the first array to be tested for a mismatch
   * @param b the second array to be tested for a
     mismatch
   * @return the index of the first mismatch between the
     two arrays,
   * otherwise {@code -1}.
   * @throws NullPointerException
   * if either array is {@code null}
   * @since 9
*/
public static int mismatch(double[] a, double[] b)
\end{minted}


\end{document}
