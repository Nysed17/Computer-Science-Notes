\documentclass[12pt, letterpaper]{article}
\usepackage[utf8]{inputenc}
\usepackage{tikz}
\usetikzlibrary{fit}
\usepackage{pgfkeys}
\usepackage{fancyvrb}
\usepackage{xcolor}
\usepackage{qtree}
\usepackage{tikz-qtree}
\usepackage{linguex}
\usepackage{multicol}
\usepackage{forest}
\usepackage{graphicx}
\usepackage{tabularx}
\usetikzlibrary{arrows}
\usepackage{amsmath}

\tikzset{
  arn_n/.style = {treenode, circle, black, draw=black,
    text width=2.5em, very thick},% arbre rouge noir, noeud rouge
  arn_r/.style = {red,
    text width=2.5em, very thick},% arbre rouge noir, noeud rouge
}

\title{Algoritmi e Strutture Dati}
\author{Alex Narder}
\date{\today}

\begin{document}
\maketitle

\tableofcontents
\newpage

\section{Contenuti del corso}
In questo corso si parlerà di:
\begin{description}

	\item[•] Algoritmi e le loro complessità; capire il comportamento asintotico
	\item[•] Ricorrenze 
	\item[•] Grafi
	\item[•] Alberi di copertura minimi
	\item[•] Problema dei cammini minimi
	\item[•] Algoritmi greedy

\end{description}

\newpage

\section{Introduzione}

Un \textbf{algoritmo} per essere utile deve essere funzionale e veloce. Quindi deve essere \textbf{ottimizzato},
nel momento in cui il tempo di esecuzione è esponenziale allora l'algoritmo non è ottimizzato.
\\
\\
Ci sono alcuni problemi per cui non c'è la speranza di trovare algoritmi efficenti.
L'obiettivo di questo corso è studiare algoritmi non troppo complicati, che quindi non saranno esponenziali.
Quando vado a misurare la complessità di un algoritmo la misuro rispetto alla dimensione dell'input
dato.
\\
In alcune situazioni è molto utile capire di che tipo di input si parla, per analizzare il problema e trarre delle conclusioni.
\\
\\
Parlando di complessità si usa sempre il \textbf{worst case}, ovvero il caso peggiore, mentre tutti gli altri casi saranno
\textbf{best case} e \textbf{avarage case}.

\section{Numeri di Fibonacci}

Problema: 
\\
Scrivere un algoritmo che restituisca in uscita i numeri di Fibonacci.
\\
\\
La sequenza di Fibonacci è un insieme di numeri, la sequenza è definita ricorsivamente;
\\

\begin{center}
   \(F_n = \)
$\left \{
\begin{array}{l}
   1,\; se \; n = 1, 2\\ 
   F_{n-1} + F_{n-2}\; se \;n >= 3\\
\end{array}
\right.$
\end{center}
\(F_n\) rappresenta l'iesimo numero di Fibonacci.
\section{Versioni Algoritmo di Fibonacci}
\subsection{Versione 1 Algoritmo di Fibonacci}

La sequenza di Fibonacci è direttamente collegata alla sezione aurea.
\\Formula di \textbf{Binet};
\\
\\\(x^2 = x + 1\)
\\\(x^2 - x - 1 = 0\)
\\\(ax^2 + bx + c = 0\) $\rightarrow$ ci\; sono\; 2\; soluzioni\; \(x_{1,2}\):\\
x1 = 1,618... che chiamiamo $\phi$
\\x2 = 0,618... che\; chiamiamo\; $\hat{\phi}$\(\)\\
\\La formula di Binet dice che per ogni n maggiore o uguale a 1 risulta
\\
\\
\(F_n = \frac{1}{\sqrt{5}} * (\phi^n - \hat{\phi}\)\(\;^n)\)
\\
\\
Dimostrare la formula di Binet: (induzione su n)\\\\
\textbf{Base} $\rightarrow$ n = 1,2
\\
\\\textbf{n = 1} $\rightarrow$ \(F_1 = \frac{1}{\sqrt{5}} * (\phi - \hat{\phi}\)\(\;)\)
\\
\\\(\frac{1}{\sqrt{5}} * (\frac{1+\sqrt{5}}{2} - \frac{1-\sqrt{5}}{2)}\) = \textbf{1}
\\
\\
\\
n = 2 $\rightarrow$ \(F_2 = 1/\sqrt{5} * (\phi^2 - \hat{\phi}\)\(^2)\)
\\
\\\(\frac{1}{\sqrt{5}} * (\frac{1+\sqrt{5}}{2} - \frac{1-\sqrt{5}}{2})\) = \textbf{1}
\\
\\
\\\textbf{Ipotesi induttiva:}
\\
\\
\textbf{def.} $\rightarrow$ \(F_n = F_{n-1} + F_{n-2}  
\\
\\F_{n-1} = \frac{1}{\sqrt{5}} * (\phi^{n-1} - \hat{\phi}\)\(\;^{n-1})\) \;\; + 
\\
\\
\(
F_{n-2} = \frac{1}{\sqrt{5}} * (\phi^{n-2} - \hat{\phi}\)\(\;^{n-2}) \;\; =\)
\\
\\
\(
\frac{1}{\sqrt{5}} * [(\phi^{n-1} + \phi^{n-2}) - (\hat{\phi}\)\(\;^{n-1} + \hat{\phi}\)\(\;^{n-2})]
\)
\begin{center}
   $\left \{
\begin{array}{l}
\phi^n = \phi^{n-1} + \phi^{n-2}\\
   \hat{\phi}\;^n = \hat{\phi}\;^{n-1} + \hat{\phi}\;^{n-2}\\
\end{array}
\right.$
\end{center}
è vera per la definizione di $\phi$ e $\hat{\phi}$
\newpage

\begin{Verbatim}[commandchars=\\\{\}]
\textcolor{red}{int} \textcolor{blue}{Fib1}(\textcolor{red}{int} n)\{
   \textcolor{green}{return} 1/sqrt(5) * (phi^n - phiˆ^n);
\}
\end{Verbatim}
facendo i calcoli l'argoritmo sembra corretto, ma al numero n = 18 il risultato è errato.

\subsection{Versione 2 Algoritmo di Fibonacci}

\begin{Verbatim}[commandchars=\\\{\}]
\textcolor{red}{int} \textcolor{blue}{Fib2}(\textcolor{red}{int} n)\{
   \textcolor{olive}{if} (n<=2) \{
      \textcolor{green}{return} 1;
   \} \textcolor{olive}{else} \textcolor{green}{return} \textcolor{blue}{Fib2}(n-1) + \textcolor{blue}{Fib2}(n-2);
\}
\end{Verbatim}
\;\;n\;\;\;\;\;\;\;\;T(n)\\
\begin{tabular}{ |c|c| } 
 \hline
 1 & 1 \\ 
 2 & 1\\ 
 3 & 2+1+1 = 4\\ 
 4 & 2+4+1 = 7\\
 \hline
\end{tabular}
\\
\\
\\n è il numero, T(n) invece è il numero della istruzioni necessarie prima di ricevere un output.
\begin{center}
      T(n) =
   $\left \{
\begin{array}{l}	1 \; se \; n = 1,2\\
	2 + T(n-1) + T(n-2) \; se \;  n >= 3\\
\end{array}
\right.$
\end{center}
\(T(n) = 2+T(n-1)+T(n-2)\) $\rightarrow$ quando \(n >= 3\)
\newpage
\textbf{Albero delle ricorsioni di n = 5:}
\\
\\
\begin{center}
\begin{tikzpicture}[level distance=1.5cm,
  level 1/.style={sibling distance=3cm},
  level 2/.style={sibling distance=1.5cm}]
 \node [arn_r]{n = 5}
	 child {node [arn_r]{n = 4}
		child {node [arn_r]{n = 3}
			child {node {n = 2}}
			child {node {n = 1}}}
		child {node {n = 2}}}
	 child {node [arn_r]{n = 3}
		child {node {n = 2}}
		child {node {n = 1}}};
\end{tikzpicture}
\end{center}
La complessità dei pallini in \textbf{rosso} è 2 mentre quella degli altri è 1.
\\
Sommando tutte le complessità otteniamo il numero di istruzioni necessarie per eseguire, in questo caso 13.
\\
\\
In \textbf{rosso} ci sono i \textbf{nodi interni}
\\
Mentre in \textbf{nero} ci sono i \textbf{nodi foglia}
\\
\\
\\$T(5) = 4*2 + 5*1$ $\rightarrow$ è\; dato\; da \;un\; numero\; di\; nodi\; interi \;(i)\; moltiplicati\; per\; 2\;
+\; il\; numero\; di\; foglie\; (f)
\\
\\
$\rightarrow$ \(T(n) = 2 * i(Tn) + 1 * f(Tn) \)
\\
\\
\newpage
\subsubsection{Proposizione 1}
Dato un generico albero di ricorsione, siamo in grado di arrivare a una formula che ci dica esattamente quanti nodi interni e foglia ci sono?
\\
\\
$\rightarrow$ Sia T(n) l'albero di ricorsione relativo alla chiamata di Fib2(n),
\\allora: f(Tn) = F(n) $\rightarrow$ infatti il numero delle foglie è l'esimo numero di Fibonacci.
\\

\includegraphics[width=0.8\linewidth]{png/1}
\subsubsection{Proposizione 2}
Sia T un albero dove i nodi interni hanno esattamente 2 figli, allora:
\\Il numero di nodi interni è sempre uguale a f(T) - 1:
\begin{center}i(T) = f(t) - 1\end{center}
Da dimostrare.
\\
Quindi con la proposizione uno e la proposizione 2 otteniamo:\\
T(n) = 2(F(n) - 1) + F(n) = 3F(n) - 2
\subsubsection{Proposizione 3}
Per ogni $n >= 6$ abbiamo che $F(n) >= 2^{n/2}$
\\
\\Da dimostrare
\\
\\
Base: n = 6,7\\
I.P.: n $>=$ 8\\
\\
\[F_n = F_{n-1} + F_{n-2}\]
\[\;\;\;\;\;\;\;\;\;\;\;\;  >= 2^{(n-1) / 2} + 2^{(n-2)/2}\]
\[.\]
\[.\]
\[.\]
\[\;\;\;\;\;\;\;\;\;\;\;\;\;\;\;\;\;\;\;  >= 2^{n/2}(\frac{1}{\sqrt{2}}+\frac{1}{2}) >= 2^{n/2}\]
\subsection{Versione 3 Algoritmo di Fibonacci}

\begin{Verbatim}
int Fib3(int n){

}
\end{Verbatim}
Complessità? T(n) è più o meno uguale a n

\subsection{Versione 4 Algoritmo di Fibonacci}

\begin{Verbatim}[commandchars=\\\{\}]
\textcolor{red}{int} \textcolor{blue}{Fib4}(\textcolor{red}{int} n)\{
   \textcolor{red}{int} a = 1, b = 2;
   \textcolor{olive}{for} (\textcolor{red}{int} i = 3 ; i <= n ; i++)\{
      c = a+b;
      a = b;
      b = c;
   \}
   \textcolor{green}{return} b;
\} 
\end{Verbatim}

\subsection{In sostanza}

tabella:\\
\\
\begin{tabularx}{0.85\textwidth} { 
  | >{\raggedright\arraybackslash}X 
  | >{\centering\arraybackslash}X 
 | >{\centering\arraybackslash}X 
  | >{\centering\arraybackslash}X | }
 \hline
   & \textbf{corretto?} & \textbf{complessità temporale} & \textbf{complessità spaziale}  \\
 \hline
 Fib1 & no & costante & costante  \\
 \hline
 Fib2  & si  & esponenziale & lineare  \\
 \hline
 Fib3  & si  & lineare  & lineare  \\
 \hline
 Fib4  & si  & lineare & lneare  \\
\hline
\end{tabularx}
\\
\\
L'algoritmo Fib4 è il più efficente.

\section{Classi asintotiche}

\subsection{Prima classe}
$O(g(n)) = \{f(n)\;|\; \exists \; c > 0$\\
$\exists \;un\; n_0 \in N \;|\; \forall n >= n_0 :$\\
$f(n) <= c * g(n)\}$
\\
\\
\includegraphics[width=0.8\linewidth]{png/2}

\subsection{Seconda classe}
$\Omega(g(n)) = \{f(n)\; | \; \exists \; c > 0$\\
$\exists \; un \; n_0 \in N \; | \; \forall n >= n_0 :$\\
$f(n) <= c * g(n)\}$\\
\\
\includegraphics[width=0.8\linewidth]{png/3}
\\
\textcolor{red}{$\frac{1}{2}n^2 - 3n = \Omega(n^2)$}
\\
\\
\[\begin{split}
   \exists  c > 0 \;\;\;\;& \exists n_0 \in N \;|\; \forall n >= n_0 :\\
      & c * n^2 <= \frac{1}{2}n^2 - 3n\\
      & c * n <= \frac{1}{2}n - 3\\
      & n(\frac{1}{2} - c) >= 3
\end{split}\]
\\
Supponendo che $\frac{1}{2} - c > 0 \;\; [c < \frac{1}{2}]$\\
si ha:
\\
\[n >=  \frac{3}{\frac{1}{2}-c} = \frac{6}{1-2c}\]
\\
\\
$\rightarrow$ scegliamo $c = \frac{1}{14}$\\\\
si ottiene: $n >= \frac{6}{1-\frac{1}{7}} = 7$

\subsection{Terza classe}

\[\begin{split}
   \Theta (g(n)) = \{f(n)\;|&\; \exists c_1 > 0,\; \exists c_2 > 0 \\
   & \exists n_0 \in N \; | \; \forall n >= n_0 :\\
   & c_1\;g(n) <= f(n) <= c_2 \; g(n)\}
\end{split}\]
\\
notazione: $f(n) = \Theta(g(n))$
\\
\\
\includegraphics[width=0.8\linewidth]{png/4}
\\
\\
\underline{proprietà:}
\[\begin{split}
   f(n) = \Theta(g(n)) \leftarrow\rightarrow &f(n) = O(g(n)) \\
   & f(n) = \Omega(g(n))
\end{split}\]
\\
quindi: $\frac{1}{2}n^2 - 3n = \theta(n^2)$
\\
\\
\textcolor{red}{$\sqrt{n+10} = \theta(\sqrt{n})$}\\
\\
\[\begin{split}
   \exists c_1,c_2 > 0 \;\;\;\; &\exists n_0 \in N \;|\; \forall >= n_0 \;:\\  
   & c_1\sqrt{n} <= \sqrt{n+10} <= c_2\sqrt{n} \leftarrow\rightarrow \\
   & c_1^2n <= n+10 <= c_2^2n
\end{split}\]
\\
\begin{itemize}
   \item[1)] $c_1^2 n <= n+10 \leftarrow\rightarrow$
      \\\\
      $(c_1^2 - 1) n <= 10$
      \\\\
      se $c_1^2 - 1 <= 0 \;\;\;(c_1^2 <= 1, c_1 <= 1)$
      \\\\
      diventa vera $\forall n >= 1 \;\;\;\;\;\rightarrow c_1 = 1$ 
   \item[2)] $n + 10 <= c_2^2n \leftarrow\rightarrow$
      \\
      \\
      $n(c_2^2 - 1) >= 10$
      \\
      \\
      se $c_2 > 1 :$
      \\
      \\
      $n >= \frac{10}{c_2^2 - 1}$
      \\
      \\
      poniamo $c_2 = \sqrt{2} (> 1)$
      \\
      si ha $n >= 10$
      \\
      quindi: \\
      \textcolor{red}{$c_1 = 1$}
      \\
      \textcolor{red}{$c_2 = \sqrt{2}$}
      \\
      \textcolor{red}{$n_0 = 10$}
\end{itemize}
\subsubsection{esempi:}

\begin{itemize}
   \item[1)]log $ n = O(n)$
\\
\includegraphics[width=0.8\linewidth]{png/5}
\item[2)]
$n \log n = O(n^2)$
\\
$\forall n >= 1 : \log n <= n \leftarrow\rightarrow $
\\$n \log n <= n^2$\\
\\
\textcolor{red}{c = 1 , $n_0 = 1$}
\item[3)]
   n!  = $O(n^n)$
      \\
      \\
      infatti: 
      \[n! = 1*2*3*...n\]
      \[<= n*n*n*...n\]
      \[= n^n\]

\end{itemize}
\subsection{Quarta classe}

$o(g(n)) = \{f(n)\;|\; \forall c > 0 , \exists n \in N$ tale che $\forall n >= n_0 : f(n) < c * g(n)\}$
\\
\\osservazione:
\\$o(g(n)) \subset O(g(n))$
\\
quindi: \\
$f(n) = o(g(n)) \rightarrow f(n) = O(g(n))$
\\
\\
\underline{proprietà:}
\\
\\
$f(n) = o(g(n)) \leftarrow\rightarrow \lim_{n\to\infty} \frac{f(n)}{g(n)} = 0$
\\
\\
\includegraphics[width=\linewidth]{png/6}

\subsection{Quinta classe}

$w(g(n)) = \{f(n)\;|\;\forall c > 0, \exists n_0 \in N $ tale che $\forall n >= n_0 : c * g(n) < f(n)\}$
\\
\\
osservazione:\\
$w(g(n)) \subset \Omega(g(n))$
\\
$f(n) = w(g(n)) \rightarrow f(n) = \Omega(g(n))$
\\
\\
\underline{proprietà:}
\\
\\
$f(n) = w(g(n)) \leftarrow\rightarrow \lim_{n\to\infty} \frac{f(n)}{g(n)} = +\infty$


\end{document}
