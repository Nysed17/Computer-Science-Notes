\documentclass[12pt, letterpaper]{article}
\usepackage[utf8]{inputenc}
\usepackage{textcomp}
\usepackage{graphicx}
\usepackage[export]{adjustbox}

\title{Basi di Dati}
\author{Alex Narder}
\date{\today}

\begin{document}

\maketitle

\section{Contenuti del corso}
\begin{itemize}
   \item[•] Sistemi per Basi di Dati
   \item[•] Modello dei dati
   \item[•] Progettazione di Basi di dati
   \item[•] Modello relazionale
   \item[•] SQL per la definizione, la manipolazione e la consultazione dei DB
   \item[•] Teoria della normalizzazione
   \item[•] Amministazione di una basi di dati
   \item[•] Programmazione di applicazioni che utilizzano basi di dati
   \item[•] Applicazione: Flask + SQLAlchemy
   \item[•] NoSQL: Modelli di rappresentazione dell’informazione diversi da quello relazionale
\end{itemize}
\newpage

\section{Introduzione}

Le \textbf{basi di dati} sono ovunque, le troviamo alla base di sistemi informativi aziendali, sistemi informativi territoriali,
applicazioni internet, basi di dati distribuite, sistemi di supporto alle decisiomi, data mining ...
\\
\\
\includegraphics[width=\linewidth]{1.png}

\section{Oranizzazioni e sistemi informativi}

• Le organizzazioni hanno bisogno di gestire le informazioni per realizzare le proprie attività.
\\
• Un sistema informativo di un'organizzazione serve per la raccolta e acquisizione, l’archiviazione, conservazione l’elaborazione, trasformazione, produzione la distribuzione, comunicazione e lo scambio delle informazioni necessarie alle attività dell’organizzazione.
\\
Sistema informativo nelle organizzazioni:\\
\includegraphics[width=\linewidth]{2}
\newpage
\section{Sistemi Informatici}

• Il \textbf{sistema informatico} è l’insieme delle tecnologie informatiche e della
comunicazione (Information and Communication Technologies, ICT) a supporto
delle attività di un’organizzazione.
\\
• Il \textbf{sistema informativo automatizzato} è quella parte del sistema informativo in cui
le informazioni sono raccolte, elaborate, archiviate e scambiate usando un sistema
informatico.
\\
\\
Sistema informatico nelle organizzazioni:\\
\\
\includegraphics[width=\linewidth]{3}
\newpage
\section{Informazioni e dati}

• Nei \textbf{}sistemi informatici le \textbf{informazioni} vengono rappresentate in modo essenziale attraverso i dati.
\\
• \textbf{Informazione}: notizia, dato o elemento che consente di avere conoscenza più o
meno esatta di fatti, situazioni, modi di essere.
\\
• \textbf{Dato}: ciò che è immediatamente presente alla conoscenza, prima di ogni
elaborazione; (in informatica) elementi di informazione costituiti da simboli che
debbono essere elaborati.

\section{Classificazione dei Sistemi Informatici}
\begin{itemize}
   \item[•] \textbf{Sistemi Informatici Operativi}:
      \begin{itemize}
         \item[-] I dati sono organizzati in BD
         \item[-] Le applicazioni si usano per svolgere le classiche \textbf{attività strutturate e ripetitive}
dell'azienda nelle aree amministrativa e finanziaria, vendite, produzione, risorse
umane ecc. (calcolo paghe, emissione fatture, magazzino, ...)
      \end{itemize}
\end{itemize}

\includegraphics[width=0.8\linewidth]{4}\\
\includegraphics[width=1.2\linewidth]{6}

\begin{itemize}
   \item[•] \textbf{Sistemi Informatici Direzionali}:
      \begin{itemize}
         \item[-] La direzione intermedia e alta necessitano di: \textbf{analisi storiche} e \textbf{produzione interattiva} di rapporti di sintesi
         \item[-] Le basi di dati operative risultano inadeguate: contengono solo \textbf{dati recenti} e le operazioni coinvolgono grandi quantità di dati o sono molto complesse e
quindi rallenterebbero in modo inaccettabile le funzioni operative 
         \item[-] I dati sono organizzati in \textbf{Data Warehouse (DW)}
         \item[-] Sono gestiti da un opportuno \textbf{sistema per analisi interattive dei dati}
         \item[-] Usano \textbf{applicazioni di business intelligente} come strumenti di supporto ai processi di controllo delle presentazioni aziendali e di decisione manageriale
      \end{itemize}
\end{itemize}
\includegraphics[width=0.9\linewidth]{5}\\
\\
\\
\includegraphics[width=1.1\linewidth]{7}

\section{Sistemi per la gestione di Basi di Dati (DBMS)}
\subsection{Baso di dati (gestite da DBMS)}
In generale una qualsiasi raccolta di informazioni permanenti gestite tramite un elaboratore elettronico è 
considerata una base di dati, ma per noi:
\\
Una \textbf{Base di dati} è una suddivisa in due categorie:
\begin{itemize}
   \item[•] \textbf{Metadati}: definiscono lo schema della BD, che descrive:
      \begin{itemize}
         \item[-] struttura dei dati
         \item[-] restrizioni sui valori ammissibili, quindi i \textbf{vincoli di integrità}
         \item[-] utenti autorizzati
      \end{itemize}
   \item[•] \textbf{Dati}: le rappresentazioni di certi fatti conformi alle definizioni dello schema, con le seguenti caratteristiche:
      \begin{itemize}
         \item[-] sono organizzati in \textbf{insiemi omogenei}, fra i quali sono definite delle \textbf{relazioni}.
            La loro struttura e le relazioni sono descritte usando il \textbf{modello dei dati }adottato. Per esempio una collezione di dati che contiene studenti, una che contiene classi, ecc. ecc.
         \item[-] sono \textbf{molti} rispetto ai metadati, e non possono essere gestiti in memoria temporanea
         \item[-] sono \textbf{permanenti}, continuano a esistere fino a quando non vengono rimossi
         \item[-] sono \textbf{utilizzabili contemporaneamente} da diversi utenti
         \item[-] sono \textbf{protetti} sia da accessi non autorizzati che da corruzione dovuta a guasti di harware o software
         \item[-] sono accessibili tramite \textbf{transazioni}, unità di lavoro atomiche che non possono avere effetti parziali, quindi devono andare tutte a buon esito, altrimenti falliscono tutte. Come se fosse una singola operazione, anche se in realtà sono più di una.
      \end{itemize}
\end{itemize}
\newpage
\subsection{Sistemi per Basi di Dati -$>$ DBMS}

\textbf{definizione:} Un \textbf{DBMS (data Base Management System)} è un sistema centralizzato o distribuito che offre opportuni linguaggi e strumenti per:\\
\begin{itemize}
   \item[-] \textbf{definire lo schema} del DB (che va definito prima di creare dati, definito attraverso il modello dei dati del DBMS, interrogabile con le stesse modalità previste per i dati)
   \item[-] scegliere le \textbf{strutture dati} per la memorizzazione dei Dati
   \item[-] \textbf{memorizzare} i dati rispettando i vincoli definiti nello schema
   \item[-] \textbf{recuperare e modificare} i dati interattivamente (tramite un linguaggio di interrogazione, anche detto \textbf{query language})
\end{itemize}

Architettura dei DBMS centralizzati:\\

\includegraphics[width=0.95\linewidth]{8}

\includegraphics[width=0.8\linewidth]{16}\\
\includegraphics[width=0.8\linewidth]{17}\\
\includegraphics[width=0.8\linewidth]{18}\\
\includegraphics[width=0.8\linewidth]{19}\\
\includegraphics[width=0.75\linewidth]{20}

\section{Funzionalità dei DBMS}
\begin{itemize}
   \item[•] Linguaggio per la \textbf{definizione dei DB (DDL)}, \textbf{data definition language}
   \item[•] Linguaggi per l'\textbf{uso dei dati (DML)}, \textbf{data manipulation language}
   \item[•] Meccanismi per il \textbf{controllo dei dati}
   \item[•] Strumenti per il \textbf{responsabile del DB}
   \item[•] Strumento per lo \textbf{sviluppo delle applicazioni}
\end{itemize}

\section{DDL: Definizione dei DB}

Per definire la base di dati si usa il DDL, la descrizione del DB è indipendente dalle applicazioni che la usano.\\
Vogliamo che il DB abbia una sua vita indipendente nonostante le applicazioni che vanno ad usarla.
\\
\\
Ci sono 3 livelli di descrizione dei dati;\\
- livello di vista logica\\
- livello logico, qui si descrive la struttuta e i vincoli della base di dati (sono raccolti nello schema logico)\\
- livello fisico\\

\includegraphics[width=0.8\linewidth]{10}
\newpage

   \subsection{Livello vista logica} Descrive come deve apparire la struttura della base di dati ad una certa
      applicazione (\textbf{schema esterno o vista}).\\
      \includegraphics[width=0.8\linewidth]{11}
   \subsection{Livello logico} Schema logico: Descrive la \textbf{struttura degli insiemi di dati e delle relazioni} fra
      questi, secondo un certo modello dei dati, senza nessun riferimento alla loro
      organizzazione fisica nella memoria permanente.\\
      \includegraphics[width=0.8\linewidth]{12}
   \subsection{Livello fisico} Descrive lo \textbf{schema fisico} o \textbf{interno}: come vanno organizzati fisicamente i dati nelle memorie permanenti, strutture dati ausiliarie per l’uso (es. indici). Un indice è quella "cosa" che mi permette di recuperare in maniera veloce un'informazione.\\
      \includegraphics[width=0.8\linewidth]{13}


\subsection{\textbf{Indipendenza fisica e indipendenza logica:\\}}
\begin{itemize}
   \item[•] \textbf{Indipendenza fisica}: i programmi applicativi non devono essere modificati in
      seguito a modifiche dell’organizzazione fisica dei dati. Quindi se io vado a cambiare qualcosa non devo modificare i programmi applicativi. \\
      \includegraphics[width=1\linewidth]{14}
   \item[•] \textbf{Indipendenza logica}: i programmi applicativi non devono essere modificati in
      seguito a modifiche dello schema logico. \\
      \includegraphics[width=1\linewidth]{15}
\end{itemize}

\section{DML: linguaggi per l'uso dei dato}

Un DBMS prevede varie modalità d’uso per soddisfare le esigenze delle diverse
\textbf{categorie di utenti} che possono accedere alla base di dati:\\
- \textbf{Utenti delle applicazioni}, nessun tipo di accesso elaborato\\
- \textbf{Utenti non programmatori}, hanno bisogno di agire in modo più evoluto rispetto agli utenti delle app.\\
- \textbf{Programmatori delle applicazioni}, utenti che hanno bisogno di operare in maniera intensiva nei DB.\\
\begin{itemize}
   \item[•] \textbf{Utenti non programmatori}:
      \begin{itemize}
         \item[-] Interfaccia grafica per accedere ai dati in modo semplice.
         \item[-] Linguaggio per interrogazione
      \end{itemize}
   \item[•] \textbf{Utenti programmatori}:
      \begin{itemize}
         \item[-] Linguaggio integrato (dati e DML): Linguaggio pensato per SQL, i comandi SQL sono controllati staticamente dal traduttore ed eseguiti dal DBMS
         \item[-] Linguaggio convenzionale + funzioni di libreria predefinita: Linguaggio convenzionale che usa delle funzioni di una libreria predefinita
per usare SQL. I comandi SQL sono stringhe passate come parametri alle
funzioni che poi vengono controllate dinamicamente dal DBMS prima di
eseguirle.
         \item[-] Linguaggio che ospita l'SQL (SQL embedded): Linguaggio convenzionale esteso con un nuovo costrutto per marcare i
            comandi SQL. Occorre un \textbf{pre-compilatore} che controlla i comandi SQL, li
sostituisce con chiamate a funzioni predefinite e genera un programma nel
linguaggio convenzionale + funzioni di libreria
      \end{itemize}
\end{itemize}

\section{Controllo dei Dati}

\subsection{DBMS: controllo dei dati}

Sono dei meccanismi offerti per garantire le seguenti proprietà:
\begin{itemize}
   \item[•] \textbf{Integrità}, mantenimento delle proprietà specificate in modo dichiarativo nello
schema (vincoli d’integrità)
   \item[•] \textbf{Sicurezza}, protezione dei dati da usi non autorizzato:
      \begin{itemize}
         \item[-] restrizione dell'accasso ai \textbf{soli utenti autorizzati}
         \item[-] \textbf{limitazione delle operazioni esegiubili}
      \end{itemize}
   \item[•] \textbf{Affidabilità}, protezione dei dati da:
      \begin{itemize}
         \item[-] \textbf{interferenze} indesiderate dovute all’accesso concorrente ai dati da parte di
più utenti
         \item[-] \textbf{malfunzionamenti hardware o software} (fallimenti di transazione, fallimenti
di sistema, disastri)
      \end{itemize}
\end{itemize}

\subsubsection{Protezione da interferenze indesiderate tra accessi concorrenti ai dati}
\begin{itemize}
   \item[•] Basterebbe impedire l'inizio di una transazione prima che un'altra inizio
   \item[•] scheduling dei singoli passi di ciascuna transazione in {T1, ..., Tn} che
garantisca che l’effetto complessivo sarebbe ottenibile eseguendo le transazioni
isolatamente in qualche ordine\\
      \includegraphics[width=\linewidth]{21}
\end{itemize}

\subsubsection{Protezione da malfunzionamenti hardware o software}
\begin{itemize}
   \item[•] \textbf{fallimenti di transazione}: dovuta a una situazione prevista dall’applicazione o a
eventi imprevisti, come la violazione di vincoli di integrità o accessi non
autorizzati
   \item[•] \textbf{fallimenti di sistema}: dovuti ad un’anomalia HW o SW dell'unità centrale o di
una periferica, che determina l'interruzione di tutte la transazioni attive e la
perdita del contenuto della memoria temporanea
   \item[•] \textbf{disastri}: danni alla memoria permanente
\end{itemize}

\subsection{DBMS: Controllo dei dati: Transazioni}

Una \textbf{transazione} è una sequenza di azioni di lettura e scrittura in
memoria permanente e di elaborazioni di dati in memoria temporanea, con le
seguenti proprietà:
\begin{itemize}
   \item[•] \textbf{Atomicità}, le transazioni che terminano prematuramente sono trattate dal sistema come se 
   non fossero mai iniziate, quindi eventuali effetti sul DB sono annullati
   \item[•] \textbf{Serializzabilità}, nel caso di esecuzioni concorrenti di più transazioni, l’effetto
complessivo è quello di una esecuzione seriale
   \item[•] \textbf{Persistenza}, Le modifiche sulla base di dati di una transazione terminata
normalmente sono permanenti, cioè non sono alterabili da eventuali
malfunzionamenti
\end{itemize}

\section{Strumenti per l'amministazione}
\subsection{DBMS: strumenti per l'amministrazione}

\begin{itemize}
   \item[•] Linguaggio per la definizione e la modifica degli schemi della base di dati
- logico, esterno, e fisico
\item[•] Strumenti per il controllo e messa a punto del funzionamento del sistema.
\item[•] Strumenti per stabilire i diritti di accesso ai dati 
\item[•] Strumenti per ripristinare la base di dati in caso di malfunzionamenti di sistemi o disastri 
\end{itemize}

\subsection{riepilogo dei vantaggi dei DBMS}

\begin{itemize}
   \item[•] indipendenza fisica e logica 
   \item[•] gestione efficiente dei dati 
   \item[•] integrità e sicurezza dei dati  
   \item[•] accessi interattivi, concorrenti e protetti da malfunzionamenti
   \item[•] amministrazione dei dati
   \item[•] riduzione dei tempi di sviluppo delle applicazioni 
\end{itemize}

\subsection{riepilogo degli svantaggi dei DBMS}

\begin{itemize}
   \item[•] Possono essere costosi e complessi da installare e mantenere in esercizio.
   \item[•] Richiedono personale qualificato (se si tratta di personale esterno, aumenta la
dipendenza da ditte esterne) 
   \item[•] Le applicazioni sviluppate possono essere trasferite con difficoltà su sistemi diversi
se vengono usati linguaggi troppo “legati” al DBMS usato
   \item[•] La riduzione dei costi della tecnologia e i possibili tipi di DBMS disponibili sul
mercato facilitano la loro diffusione
\end{itemize}

\section{Progettazione e Modellazione}

\begin{itemize}
   \item[•] Progettare una base di dati significa progettare la:
      \begin{itemize}
         \item[-] struttura dei dati
         \item[-] applicazioni
      \end{itemize}
   \item[•] La progettazione della struttura dei dati è l'attività fondamentale
   \item[•] Richiede di specificare un modello della realtà di interesse (\textbf{universo del
   discorso}) quanto più possibile fedele
   \item[•] Per questo ci concentreremo sulla modellazione:
      \begin{itemize}
         \item[-] cosa significa definire un modello?
         \item[-] cosa si modella?
         \item[-] come si modella (quale formalismo)?
      \end{itemize}
\end{itemize}
\newpage
\section{Modello dei Dati e Progettazione}
\subsection{Modelli Informatici}
Un \textbf{modello astratto} è la rappresentazione formale di idee e
conoscenze relative a un fenomeno.
\\
\\
• Aspetti di un modello:
\begin{itemize}
   \item[-] il modello è la \textbf{rappresentazione di certi fatti};
   \item[-] la rappresentazione è data con un \textbf{linguaggio formale};
   \item[-] il modello è il risultato di un \textbf{processo di interpretazione}, guidato dalle idee
e conoscenze possedute dal soggetto che interpreta.
\end{itemize}

\subsection{Progettazione di un DB}

\includegraphics[width=0.7\linewidth]{22}
\\
Ogni una di queste fasi è incentrata sulla modellazione.

\newpage

\section{Modellazione concettuale}

\subsection{Aspetti del problema}

\begin{itemize}
   \item[•] Quale conoscenza del dominio nel discorso si rappresenta?\\
      \textbf{()aspetto ontologico)}
   \item[•] Con quali meccanismi di astrazione si modella?\\
      \textbf{(aspetto logico)}
   \item[•] Con quale linguaggio formale si definisce il modello?\\
      \textbf{(aspetto linguistico)}
   \item[•] Come si procede per costruire un modello?\\
      \textbf{(aspetto pragmatico)}
\end{itemize}

\section{Cosa si modella?}

\begin{itemize}
   \item[•] \textbf{Conoscenza concreta}: I fatti
   \item[•] \textbf{Conoscenza astratta}: Struttura e vincoli sulla conoscenza concreta 
   \item[•] \textbf{Conoscenza procedurale}: Le operazioni di base e le operazioni degli utenti
   \item[•] \textbf{Comunicazioni}: Come si comunicherà con il sistema informatico 
\end{itemize}

\subsection{Conoscenza concreta}

• Fatti specifici che si vogliono rappresentare:
- \textbf{le entità} con le loro \textbf{proprietà}
- le \textbf{collezioni} di identità omogenee
- le \textbf{associazioni} fra entità

\subsubsection{entità e proprietà}

• Le \textbf{entità} sono ciò di cui interessa rappresentare alcuni fatti (\textbf{o proprietà}):\\
oggetti concreti, oggetti astratti ed eventi.
\\
Come ad esempio un libro, una descrizione bibliografica o un prestito.
\\
\\
• Le \textbf{proprietà} si distinguono dalle entità poichè sono fatti che interessano solo perchè descrivono caratteristiche di determinata identità
\\
come ad esempio: indirizzo che interessa solo in quanto indirizzo di un utente.
\\
\\
nota: un'entità non coincide con i valori delle sue proprietà.
\\
\\
• Una \textbf{proprietà} è una coppia $<$Attributo, valore di un certo tipo$>$.
\\
\\
• Classificazione delle proprietà:\\
- \textbf{atomica o strutturata}\\
- \textbf{univoca / multivalore}\\
- \textbf{totale / parziale}
\\
\\
• Esempi:\\
- nome (atomica, univoca, totale)\\
- residenza = [indirizzo, cap, città] (strutturata)\\
- recapiti telefonici (multivalore, parziale)

\subsubsection{Collezioni di entità}

• \textbf{Tipi di entità}: ogni entità ha un \textbf{tipo} che ne specifica la natura (identifica caratteristiche: proprietà e dominio relativo).
\\
\\
es: Antonio ha tipo \textbf{Persona} con proprietà:\\
$-\;Nome\;:\; string$\\
$-\;Indirizzo\;:\; string$\\
\\
es2:\\
\includegraphics[width=\linewidth]{23}
\\
• \textbf{Collezione (classe)}: un insieme variabile nel tempo di entità omogenee, ovvero dello stesso tipo.
\\
\\
es: \textbf{Studenti}: insieme di tutti gli studenti nel dominio del discorso 
\\
\\
es2:\\
\includegraphics[width=1.05\linewidth]{24}

\subsubsection{Scelta delle entità e delle proprietà}

Certi fatti possono essere interpretati come proprietà in certi contesti e come entità in altri, per esempio:\\
\\
• \textbf{Descrizione bibliografica} con proprietà; \textbf{Autori, Titolo, Editore, Anno}\\
\\oppure\\
\\
• \textbf{Autore} con proprietà \textbf{Nome, Nazionalità, AnnoNascita, }...\\
\\• \textbf{Editore} con proprietà \textbf{Nome, Indirizzo, e-mail,} ...\\
\\• \textbf{Descrizione} biblioteca con proprietà \textbf{Titolo, Anno, ... }

\subsubsection{Gerarchie}

• Spesso le collezioni di enrità sono organizzate in una gerarchia di \textbf{specializzazione/generalizzazione} (si parla anche di 
\textbf{sottoclassi} e \textbf{superclassi}).
\\
\\
• Es: nel DB della biblioteca la collezione degli $Utenti$ può essere considerata una 
generalizzazione di $Studenti$ e $Docenti$
\\
\\
• Due importanti caratteristiche delle gerarchie:
\\- \textbf{ereditarietà} proprietà 
\\-\textbf{ inclusione}: se la collezione C1 specializza C2, gli elementi di C1 sono un sottoinsieme degli elementi di C2.
\\
\\
\underline{Problema: scelta delle sottoclassi $\rightarrow$}\\
\includegraphics[width=\linewidth]{25}

\subsubsection{Associazioni}

\begin{itemize}
   \item[•] Un'\textbf{istanza di associazione} è un fatto che correla due o più entità, 
      stabilendo un legame logico tra di loro.
      \begin{itemize}
         \item[-] la descrizione bibliografica con titolo "$Basi\;di\;Dati$" riguarda il documento 
            con collocazione "$D3-55-2$"
         \item[-] l'utente "$Tizio$" ha in prestito una copia della "$Divina Commedia$"
      \end{itemize}
   \item[•] Un'\textbf{associazione} R(X,Y) fra due collezioni di entità X e Y è un insieme di istanze 
      di associazione tra elementi di X e Y, che varia in generale nel tempo. Il prodotto cartesiano (X * Y) è 
      detto \textbf{dominio} dell'associazione.
   \item[•] esempio:
      \\
      \\
      \includegraphics[width=1.05\linewidth]{26}
   \item[•] Tipi di associazione: Un'associazione è caratterizzata dalle seguenti proprietà strutturali:
      \begin{itemize}
         \item[-] \textbf{molteplicità} (o cardinalità)
         \item[-] \textbf{totalità}
      \end{itemize}
\end{itemize}

\subsubsection{Tipi di associazione: Molteplicità}
\begin{itemize}
   \item[•] \textbf{Vincolo di univocità:}
      \begin{itemize}
         \item[-] Un'associazione R(X,Y) è \textbf{univoca da X a Y} se per ogni elemento x di X
         esiste al più un elemento di Y che è associato ad x; se non vale questo
            vincolo, l’associazione è \textbf{multivalore da X a Y}.
      \end{itemize}
   \item[•] \textbf{Cardinalità:}
      \begin{itemize}
         \item[-] \textbf{R(X,Y) è (1:N)} se essa è \underline{multivalore} da X a Y ed \underline{univoca} da Y a X
         \item[-] \textbf{R(X,Y) è (N:1)} se essa è \underline{univoca} da X a Y e \underline{multivalore} da Y a X
         \item[-] \textbf{R(X,Y) è (N:M)} se essa è \underline{multivalore} da X a Y e \underline{multivalore} da Y a X
         \item[-] \textbf{R(X,Y) è (1:1)} se essa è \underline{univoca} su da X a Y e \underline{univoca} da Y a X
      \end{itemize}
   \item[•] esempi:\\
      \includegraphics[width=\linewidth]{27}
\end{itemize}

\subsubsection{Tipi di associazione: Totalità}

\begin{itemize}
   \item[•] \textbf{Vincolo di totalità}: Un'associazione R(X,Y) è \textbf{totale} da X a Y se per ogni elemento x di X 
      esiste almeno un elemento di Y che è associato ad x; se non vale questo vincolo, l'associazione è \textbf{parziale} da X a Y.
      \\
      \\
      Esempio: $Insegna(Professori,\;Corsi)$ è totale su $Corsi$ in quanto non può esistere un corso 
      del piano di studi senza il corrispondente docente che lo tiene, parziale su $Professori$, in quanto un professore potrebbe non tenere corsi.
   \item[•] esempi:\\
      \includegraphics[width=0.8\linewidth]{28}
\end{itemize}

\subsection{Conoscenza Astratta}

Fatti generali che descrivono:

\begin{itemize}
   \item[•] la \textbf{struttura della conoscenza concreta}
      \begin{itemize}
         \item[-] \textbf{collezioni}: nomi, tipo degli elementi (nome, dominio, caratteristiche delle proprieta)
         \item[-] \textbf{associazioni}: nomi, collezioni correlate, proprietà strutturali 
      \end{itemize}
   \item[•] restrizioni sui \textbf{valori possibili} della conoscenza concreta e sui modi in cui esse 
      possono \textbf{evolvere} nel tempo (vincoli di integrità)
   \item[•] regole per \textbf{derivare} nuovi fatti da altri noti 
\end{itemize}

\subsubsection{Vincoli di Integrità}

Ce ne sono due tipi:

\begin{itemize}
   \item[•] \textbf{Vincoli di integrità statici}: definiscono delle condizioni sui valori della conoscenza concreta che devono essere soddisfatte 
      indipendentemente da come evolte l'universo del discorso 
      \\
      \\Es: $Stipendio$ deve essere positivo, $Matricola$ è una chiave, ...
   \item[•] \textbf{Vincoli di integrità dinamici}: definiscono delle condizioni sul modo in cui la 
      conoscenza concreta può evolvere nel tempo 
      \\
      \\Es: $DataNascita$ non può cambiare, uno studente iscritto ad un corso di laurea non può iscriversi nuovamente, ...
\end{itemize}

\subsubsection{Fatti derivabili}

Fatti derivabili:

\begin{itemize}
   \item[•] L'età di una persona, ricavabile per differenza fra l'anno attuale e il suo anno di nascita 
   \item[•] La media dei voti degli esami superati da uno studente, derivabile attraverso dei calcoli 
\end{itemize}

\section{Come si modella?}

\subsection{Modello dei dati a oggetti}

\begin{itemize}
   \item[•] Un \textbf{modello dei dati} è un insieme di meccanismi di astrazione per descrivere la struttura della conoscenza concreta (\textbf{schema})
   \item[•] Uno schema verrà dato usando una notazione grafica, variante dei cosiddetti
      \textbf{diagrammi ER} (Entità-Relazione)
   \item[•] Nozioni fondamentali:
      \begin{itemize}
         \item[-] Oggetto, Tipo di oggetto, Classe 
         \item[-] Ereditarietà, Gerarchia fra tipi, Gerarchia fra classi 
      \end{itemize}
\end{itemize}

\subsection{Questioni terminologiche}

\includegraphics[width=\linewidth]{29}

\newpage

\subsection{Oggetto}

\begin{itemize}
   \item[•] Ad ogni \textbf{entità} del dominio corrisponde un \textbf{oggetto} del modello informatico 
   \item[•] Un oggetto è un'entità software con \textbf{stato, comportamente e identità}
      \begin{itemize}
         \item[-] Lo \textbf{stato} è modellato da un insieme di \textbf{costanti o variabili} con valori di qualsiasi 
            complessità
         \item[-] Il \textbf{comportamento} è modellato da un insieme di procedure locali (con parametri) chiamate \textbf{metodi} 
         \item[-] L'\textbf{identità} è associata all'oggetto dalla creazione e non viene modificata da aggiornamenti dello stato 
      \end{itemize}
   \item[•] Un oggetto può rispondere a richieste, dette \textbf{messaggi}, restituendo valori memorizzati 
      nello stato o calcolati con una procedura locale 
\end{itemize}

\subsection{Tipo Oggetto}

\begin{itemize}
   \item[•] Il primo passo nella costruzione di un modello consiste nella \textbf{classificazione}
      \textbf{delle entità del dominio} con la identificazione di collezioni omogenee e la
      definizione dei \textbf{tipi degli oggetti} che le rappresentano
   \item[•] Un \textbf{tipo oggetto} definisce:
      \begin{itemize}
         \item[-] componenti dello stato 
         \item[-] metodi
      \end{itemize}
   \item[•] Noi useremo solo un paio di concetti elementari del modello a oggetti
   \item[•] Oggetto:
      \begin{itemize}
         \item[-] identità
         \item[-] stato (insieme di attributi)
      \end{itemize}
   \item[•] no metodi, no incapsulamento, ecc.
\end{itemize}

\subsection{Classi}

Una classe è un insieme di oggetti dello stesso tipo, modificabile con operatori per
includere o estrarre elementi dall’insieme, associabile a vincoli di integrità.
\\
\includegraphics[width=\linewidth]{30}
Una classe \textbf{$Persone$} a diversi livelli di specifica
\\
\begin{itemize}
   \item[•] I tipi degli attributi possono essere:
      \begin{itemize}
         \item[-] \textbf{primitivi} (int, bool, string, date, real)
         \item[-] \textbf{non primitivi}
      \end{itemize}
   \item[•] \textbf{Tipi non primitivi:} ottenuti applicando i seguenti operatori ad altri tipi:
      \begin{itemize}
         \item[-] \textbf{tipo record}; [A1;T1; ... , An:Tn]
         \item[-] \textbf{tipo enumerazione}; [Val1; ... ; Valn]
         \item[-] \textbf{tipo sequenza}; seq T
      \end{itemize}
\end{itemize}

\subsection{Associazioni}

\begin{itemize}
   \item[•] Le associazioni si modellano con un costrutto apposito
   \item[•] Le associazioni possono avere delle \textbf{proprietà}
   \item[•] Le associazioni possono essere \textbf{ricorsive}
   \item[•] Associazioni n-arie
\end{itemize}
\includegraphics[width=0.6\linewidth]{31}

\subsection{Associazioni n-arie}

\includegraphics[width=\linewidth]{32}

\subsection{Descrizione di un caso}

Si vogliono modellare alcuni fatti riguardanti una biblioteca universitaria:

\begin{itemize}
   \item[•] le \textbf{descrizioni bibliografiche} dei libri, opere con un solo volume ...
   \item[•] i \textbf{termini del thesaurus} (parole chiave)
   \item[•] le \textbf{copie dei libri} disponibili che corrispondono ad una descrizione bibliografica
   \item[•] gli \textbf{autori} dei libri 
   \item[•] gli \textbf{utenti} della biblioteca
   \item[•] i \textbf{prestiti} in corso 
\end{itemize}
Le Descrizioni Bibliografiche dei libri a volume unico, sia già disponibili che quelli
ordinati ma non ancora consegnati alla biblioteca, sono caratterizzati dal Codice
ISBN (International Standard Book Number) che le identifica, titolo del libro, autori,
editore, anno di pubblicazione e termini che le descrivono.\\
Degli autori dei libri interessano il codice fiscale, il nome e cognome, la nazionalità
e la data di nascita.\\
I libri, disponibili in una o più copie ognuna identificata da un Codice, una stringa
con la loro collocazione e numero.\\
Quando un utente prende un libro in prestito, si registrano i dati dell'utente, se non
sono già presenti, la data del prestito e la data di restituzione. Di un utente
interessano il codice fiscale, il nome, il cognome, l'indirizzo e i recapiti telefonici.
Un utente può avere più libri in prestito. I dati su un prestito interessano fino al
momento della restituzione del libro.\\
\\
Il thesaurus è un insieme di termini, e di associazioni fra di loro, che costituiscono il
lessico specialistico da usare per descrivere il contenuto dei libri.\\
Fra i termini del thesaurus interessano le seguenti relazioni, fra le tante possibili:
\\
\begin{itemize}
   \item[•] \textbf{Preferenza}, per rimandi da termini standard a termini non standard e viceversa.
            Per esempio:
      \begin{itemize}
         \item[-] Elaboratore \textbf{Standard (vedi)} Calcolatore;
         \item[-] Calcolatore \textbf{Sinonimi (UsatoPer)} Elaboratore, Calcolatore, Stazione di lavoro 
      \end{itemize}
   \item[•] \textbf{Gerarchia}, per mettere in evidenza il rapporto specificità-generalità tra due
            termini. Per esempio:
      \begin{itemize}
         \item[-] Felino \textbf{PiùSpecifico} Gatto Leone Tigre;
         \item[-] Gatto \textbf{PiùGenerale} Felino
      \end{itemize}
\end{itemize}

\underline{\textbf{Esempio di Biblioteca:}}
\\
\\
\includegraphics[width=1.1\linewidth]{33}

\subsection{Gerarchia tra tipi oggetto}

\begin{itemize}
   \item[•] Relazione di \textbf{sottotipo <=}, tra i tipi oggetto 
   \item[•] Se \textbf{t è sottotipo di t' (t <= t')}
      \begin{itemize}
         \item[-] gli elementi di tipo t possono essere usati in ogni contesto 
            in cui possano apparire elementi di tipo t' (\textbf{sostitutività})
      \end{itemize}
   \item[•] In particolare:
      \begin{itemize}
         \item[-] gli elementi di t hanno tutti gli attributi degli elementi di t' 
         \item[-] per ogni attributo A in t', il suo tipo in t è sottotipo del suo tipo in t'
      \end{itemize}
\end{itemize}

\subsection{Ereditarietà}

\begin{itemize}
   \item[•] \textbf{Ereditarietà (inheritance)}: permette di definire un tipo oggetto a partire da un altro "per differenza"
      \begin{itemize}
         \item[-] aggiunta di attributi 
         \item[-] ridefinizione di attributi esistenti
      \end{itemize}
   \item[•] Normalmente l'ereditarietà tra tipo si usa solo per definire sottotipi (\textbf{ereditarietà stretta}); in questo caso:
      \begin{itemize}
         \item[-] gli attributi possono essere aggiunti 
         \item[-] gli attributi possono essere ridefiniti solo specializzandone il tipo 
      \end{itemize}
\end{itemize}

\underline{tipi definiti per ereditarietà:}
\\
\\
\includegraphics[width=1\linewidth]{34}

\subsection{Gerarchia tra classi}

Fra classi può essere definita una relazione di \textbf{sottoclasse} (talvolta della sottoinsieme), con le seguenti proprietà:\\
\\• riflessiva, antisimmetrica e transitiva (ordine parziale).\\
\\• Se C è sotoinsieme di B, allora le entità in C sono un sottoinsieme delle entità in B (\textbf{vincolo estensionale}).\\
\\• Se C è sottoclasse di B, allora il tipo delle entità in C è sottotipi del tipo degli elementi in B (\textbf{vincolo estensionale}).
\\
\\
\underline{esempio:}
\\
\includegraphics[width=1\linewidth]{35}

\subsection{Vincoli su sottoclassi}

Ci sonodiversi tipi di vincoli:
\\
• \textbf{Vincolo di disgiunzione}:
\\
\includegraphics[width=1\linewidth]{36}
\\
• \textbf{Vincolo di copertura}
\\
\includegraphics[width=1\linewidth]{37}

\newpage

\underline{esempi:}
\\
\includegraphics[width=1\linewidth]{38}

\subsection{Come funzionano le frecce}

\begin{itemize}
   \item[•] \textbf{Freccia singola}; una corrispondenza. 
   \item[•] \textbf{Freccia multipla}; più corrispondenze.
   \item[•] \textbf{Barretta prima delle freccie}; potrebbe non esserci corrispondenza.
\end{itemize}
Quando le frecce sono da entrambi i lati vuol dire che da entrambe le vie ci possono essere 
più corrispondenze. Ad esempio un conto corrente può avere più intestatari, e una persona può avere più conti corrente.
\\
\\
Per corrispondenza si intende che ci ha a che fare.

\newpage

\subsection{Modello concettuale}

\begin{itemize}
   \item[•] Gli utenti possono essere studenti o docenti. Di uno studente interessa anche la matricola e di un docente anche il telefono dell’ufficio.
   \item[•] Alcuni libri sono per la sola consultazione e possono essere presi in prestito per un numero prefissato di giorni solo dagli utenti che sono docenti.
\end{itemize}

\begin{center}\includegraphics[width=1.2\linewidth]{39}\end{center}

\newpage

\subsubsection{Esempi di esercizi:}

\begin{itemize}
   \item[1)] Si vogliono rappresentare informazioni relative a una catena di negozi di abbigliamento. Un negozio ha un indirizzo, un numero di telefono e l’orario di apertura costituito dai giorni della settimana e dalle ore di apertura e chiusura. I vestiti sono identificati da un codice e hanno un modello e una taglia che è espressa con XS, S, M, L o XL se è un vestito per adulti, dall’altezza minima e massima se è per bambini. I modelli sono caratterizzati da un nome, una descrizione, un insieme di colori, un prezzo e dai negozi in cui questi modelli sono disponibili. Inoltre si vogliono memorizzare informazioni relative ai clienti: il nome, il cognome e il numero della carta fedeltà. I clienti possono fare diversi acquisti e si vuole conoscere i vestiti acquistati, il giorno, l’ora, la modalità di pagamento dell’acquisto e il negozio in cui si è effettuato l’acquisto. 
\end{itemize}

\includegraphics[width=0.9\linewidth]{43}

\subsection{Descrittore di classe con vincoli}

\includegraphics[width=\linewidth]{40}
\\
$<<$K$>>$\textbf{ chiave}: sottoinsieme minimale di attributi che identifica l'oggetto 
\\
$<<$NOT NULL$>>$ \textbf{totalità}
\\
\\
$self.Nome$ = attributo $Nome$ dell'oggetto stesso
\\
\\
In $Esami$ potremmo usare $self.HaSostenuto.Matricola$

\subsection{Formalismi grafici}

\includegraphics[width=\linewidth]{41}

\section{Progettazione di Data Base}

\subsection{Progettazione concettuale di schemi}

\begin{itemize}
   \item[•] Identificare le classi 
   \item[•] Identificare le associazini con relative proprietà strutturali 
   \item[•] Identificare gli attributi delle classi e associazioni e i loro tipi 
   \item[•] Individuare le sottoclassi 
   \item[•] Individuare le generalizzazioni 
   \item[•] Vincoli di integrità
\end{itemize}

\subsection{Esercizio: Segreteria studenti}

La segreteria si occupa di pratiche relative agli studentei del corso di laurea 
in informatica, e in particolare delle richieste di trasferimento da altri corsi di laurea.
\\
\\
Una \textbf{Domanda di trasferimento} viene trasmessa al CCS e determina l’istruzione di
una pratica di trasferimento. La pratica viene infine evasa da una delibera che
indica quali esami sono stati convalidati. Un esame di un corso esterno viene
convalidato come esame interno, e la convalida di un esame può essere piena o
previo colloquio integrativo.
\\
\\
Di una domanda di trasferimento interessano il numero di protocollo (che la
identifica), il nome e recapito dello studente che la presenta, Università, Facoltà,
Corso di laurea di provenienza, data di presentazione, elenco dei corsi sostenuti.
\\
\\
Di una pratica di trasferimento interessano la domanda di trasferimento a cui si
riferisce, il numero progressivo che la identifica, annotazioni eventuali, l’eventuale
delibera relativa.
\\
\\
Di una delibera interessano la pratica relativa, gli esami convalidati, annotazioni
eventuali, data e numero del verbale del CCS (assenti quando la delibera è in stato
di bozza).
\\
\\
Di un corso esterno interessano nome, Università, Facoltà, Corso di laurea (non
necessariamente quelli di provenienza dello studente), Anno Accademico di
superamento.
\\
\\
Di un corso interno interessano nome e numero di crediti.
\\
\\
Di una convalida d’esame interessa corso interno, corso esterno e se l’esame è
convalidato in modo pieno o previo colloquio.
Per alcuni corsi esterni esistono convalide tipiche, che indicano corso interno
corrispondente e modalità di convalida.
\\
\\
\underline{Disegnamo lo schema iniziale con sottocalssi e studanti:}
\\
\\
\includegraphics[width=\linewidth]{44}

\newpage

\underline{Tipi degli attributi:}
\\
\\
\includegraphics[width=\linewidth]{45}
\\
Ora che abbiamo un'idea generale di come debba essere fatto lo schema con vincoli 
di integrità possiamo andare a farlo:

\newpage

\underline{Schema finale: vincoli di integrità}
\\
\\
\\
\includegraphics[width=1.4\textwidth, center]{46}

\subsection{Integrazione di schemi}

Per i grandi database accade che si costruiscano sottoschemi settoriali che vanno 
poi integrati.
\\
\\
• \textbf{Soluzione dei conflitti} (nome, tipo, proprietà strutturali, vincoli di integrità)\\
• \textbf{Fusione degli schemi}\\
• \textbf{Analisi delle proprietà interschema}
\\
\\
\underline{esempio:}
\\
• In una biblioteca:
\\
\includegraphics[width=12cm]{47}
\\
• Risoluzione dei conflitti di nome, di tipo e di vincoli di i.
\\
dobbiamo uniformare gli schemi, nello schema B $Editore$ diventa una entità, $Nome$ diventa Titolo, $Descrittori$ diventa Argomenti
\\
\\
\includegraphics[width=12cm]{48}
\\
\\
• Fondere gli schemi:
\\
\\
\includegraphics[width=12cm]{49}

\newpage

• Proprietà interschema 
\\
\\
\includegraphics[width=12cm]{50}

\section{Progettazione Logica: Il Modello Relazionale}

• Il modello dei dati relazionale
\\
\\
• Trasformazione dal modello concettuale ad oggetti al modello logico
relazionale
\\
\\
• Algebra relazionale

\section{Il Modello Relazionale}

\subsection{Intuizione}

\begin{itemize}
   \item[•] \textbf{Collezioni} come \textbf{relazioni (tabelle)}
      \\
      \includegraphics[width=12cm]{51}
   \item[•] \textbf{Associazioni} tramite \textbf{chiavi}
      \\
      \includegraphics[width=12cm]{52}
\end{itemize}



\end{document}
